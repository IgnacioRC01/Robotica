\section{Valenzuela, Horacio}
En este curso de Robótica aprendimos a usar herramientas como GitHub, Visual Studio, MATLAB, SolidWorks, ROS Gazebo y Ubuntu para poder simular un robot. Aunque logramos buenos resultados al final, el camino no fue fácil. Uno de los principales problemas fue entender algunos temas complejos de robótica, especialmente al trabajar con códigos en MATLAB que venían de fórmulas matemáticas. Al principio fue muy difícil interpretarlos, pero con la práctica y los ejercicios con diferentes robots, poco a poco fui entendiendo mejor.

También tuvimos algunos problemas técnicos, sobre todo con MATLAB, ya que en ocasiones los programas no funcionaban bien o daban errores. Aunque esto fue molesto, lo pudimos resolver con paciencia y buscando soluciones.

Algo que me gustó mucho fue trabajar con GitHub. Me pareció una herramienta muy útil y práctica para organizar el código y trabajar de forma más ordenada. Además, el uso de LaTeX para hacer reportes me pareció muy bueno, ya que permite presentar trabajos más completos y profesionales.

Lo que más disfruté fue cuando todo el trabajo empezó a tener sentido y pudimos ver al robot funcionando en la simulación. Ver que los códigos, los cálculos de fuerzas, los sistemas de referencia y todo lo demás daban resultado fue muy motivador. A pesar de los problemas, todo valió la pena.

Estoy agradecido de haber aprendido a usar todas estas herramientas. Me ayudaron mucho en este curso y estoy seguro de que me servirán también en proyectos futuros.