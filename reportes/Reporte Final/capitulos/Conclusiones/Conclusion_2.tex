\section{Ortega, Jason}
A lo largo del curso de Robótica, tuve la oportunidad de aprender y aplicar diversas herramientas como GitHub, Visual Studio, MATLAB, SolidWorks, ROS, Gazebo y LaTeX, cada una con su grado de complejidad y utilidad. GitHub y Visual Studio destacaron como recursos clave para facilitar el trabajo en equipo, permitiéndonos organizar el código, colaborar de manera más estructurada y comprender mejor el flujo de trabajo en proyectos compartidos. Aunque LaTeX representó un reto para mí al inicio, terminé enfocando mis esfuerzos en la programación, donde me sentí más seguro Yal final pude comprenderlo y tener un dominio hasta cierto punto.

Enfrenté varios desafíos, como errores técnicos en MATLAB y la dificultad inicial para interpretar códigos derivados de fórmulas matemáticas. Sin embargo, la práctica constante y el trabajo colaborativo nos ayudaron a superar esas barreras. Uno de los momentos más gratificantes fue ver cómo todos los elementos desde los cálculos hasta los sistemas de referencia se integraban para dar vida a la simulación del robot. A pesar de las complicaciones, la experiencia fue enriquecedora y me brindó habilidades valiosas que sin duda aplicaré en futuros proyectos.

Utilizar Github y Visual Studio de la forma corecta facilita cualquier trabajo en equipo.