\section{Características del Robot} \label{sec:caracteristicas_del_robot}


En esta sección se describe la estructura cinemática del robot utilizando el método de Denavit-Hartenberg (DH). Este método permite representar la geometría del manipulador mediante cuatro parámetros para cada articulación o eslabón, facilitando la formulación de la cinemática directa e inversa.

\subsection*{Descripción de los parámetros}

Los parámetros utilizados son los siguientes:

\begin{itemize}
	\item \textbf{\(\theta\)}: Ángulo de rotación alrededor del eje \(z\) (solo para articulaciones rotacionales).
	\item \textbf{\(d\)}: Desplazamiento a lo largo del eje \(z\) (solo para articulaciones prismáticas).
	\item \textbf{\(a\)}: Longitud del eslabón o distancia entre los ejes \(z\), medida sobre el eje \(x\).
	\item \textbf{\(\alpha\)}: Ángulo de torsión entre ejes \(z\), medido sobre el eje \(x\).
	\item \textbf{Tipo}: Tipo de articulación. 'r' indica rotacional y 'p' prismática.
	\item \textbf{\(q_{\text{min}}\) y \(q_{\text{max}}\)}: Límites inferiores y superiores del rango de movimiento de cada articulación.
	\item \textbf{\(\dot{q}_{\text{max}}\)}: Velocidad máxima permitida para la articulación.
	\item \textbf{\(\ddot{q}_{\text{max}}\)}: Aceleración máxima permitida.
	\item \textbf{\(\tau\)}: Torque máximo para articulaciones rotacionales o fuerza máxima para articulaciones prismáticas.
	\item \textbf{\(\mu_s\)}: Coeficiente de fricción estática.
	\item \textbf{\(\mu_k\)}: Coeficiente de fricción dinámica.
\end{itemize}

\subsection*{Tabla de parámetros DH y límites dinámicos}

A continuación, se presenta la tabla DH del robot utilizado en este trabajo, junto con sus límites dinámicos para cada articulación:

\begin{table}[H]
	\centering
	\caption{Parámetros DH y límites dinámicos del robot}
	\begin{tabular}{|c|c|c|c|c|c|c|c|c|c|c|c|c|}
		\hline
		\textbf{N} & \(\theta\) & \(d\) & \(a\) & \(\alpha\) & Tipo & \(q_{\text{min}}\) & \(q_{\text{max}}\) & \(\dot{q}_{\text{max}}\) & \(\ddot{q}_{\text{max}}\) & \(\tau\) & \(\mu_s\) & \(\mu_k\) \\
		\hline
		1 & 0 & 7 & 3 & 0 & r & -90 & 90 & 180 & 360 & 8  & 0.1 & 0.2 \\
		2 & 0 & 0 & 3 & -90 & r & -90 & 90 & 180 & 360 & 50 & 0.1 & 0.2 \\
		3 & 0 & 2 & 2 & -90 & r & -90 & 90 & 180 & 360 & 30 & 0.1 & 0.2 \\
		4 & 0 & 5 & 0 & 0 & p & 3 & 7 & 1 & 2 & 10 & 0.1 & 0.2 \\
		\hline
	\end{tabular}
	\label{tab:parametros_dh_robot}
\end{table}

