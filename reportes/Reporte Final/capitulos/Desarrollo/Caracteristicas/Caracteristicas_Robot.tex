\section{Características del Robot} \label{sec:caracteristicas_del_robot}


En esta sección se describe la estructura cinemática del robot utilizando el método de Denavit-Hartenberg (DH). Este método permite representar la geometría del manipulador mediante cuatro parámetros para cada articulación o eslabón, facilitando la formulación de la cinemática directa e inversa.

\subsection*{Descripción de los parámetros}

Los parámetros utilizados son los siguientes:

\begin{itemize}
	\item \textbf{\(\theta\)}: Ángulo de rotación alrededor del eje \(z\) (solo para articulaciones rotacionales).
	\item \textbf{\(d\)}: Desplazamiento a lo largo del eje \(z\) (solo para articulaciones prismáticas).
	\item \textbf{\(a\)}: Longitud del eslabón o distancia entre los ejes \(z\), medida sobre el eje \(x\).
	\item \textbf{\(\alpha\)}: Ángulo de torsión entre ejes \(z\), medido sobre el eje \(x\).
	\item \textbf{Tipo}: Tipo de articulación. 'r' indica rotacional y 'p' prismática.
	\item \textbf{\(q_{\text{min}}\) y \(q_{\text{max}}\)}: Límites inferiores y superiores del rango de movimiento de cada articulación.
	\item \textbf{\(\dot{q}_{\text{max}}\)}: Velocidad máxima permitida para la articulación.
	\item \textbf{\(\ddot{q}_{\text{max}}\)}: Aceleración máxima permitida.
	\item \textbf{\(\tau\)}: Torque máximo para articulaciones rotacionales o fuerza máxima para articulaciones prismáticas.
	\item \textbf{\(\mu_s\)}: Coeficiente de fricción estática.
	\item \textbf{\(\mu_k\)}: Coeficiente de fricción dinámica.
\end{itemize}

\subsection*{Tabla de parámetros DH y límites dinámicos}

	
	\section*{Parámetros Denavit-Hartenberg del Robot \texttt{robot0}}
	
	La siguiente tabla representa los parámetros Denavit-Hartenberg modificados del robot denominado \texttt{robot0}, el cual tiene 6 articulaciones rotacionales.
	
	\begin{table}[h!]
		\centering
		\begin{tabular}{ccccccccccc}
			\toprule
			\textbf{$\theta$} & \textbf{$d$} & \textbf{$a$} & \textbf{$\alpha$} & \textbf{tipo} & \textbf{min (°)} & \textbf{max (°)} & \textbf{$\dot{q}_{\text{max}}$ (°/s)} & \textbf{$\ddot{q}_{\text{max}}$ (°/s²)} \\
			\midrule
			2   & 39.91 & 0     & -90 & r & -90 & 90 & 180 & 360 \\
			-90 & 0     & 44.8  & 0   & r & -90 & 90 & 180 & 360 \\
			0   & 0     & 4.197 & -90 & r & -90 & 90 & 180 & 360 \\
			0   & 45.1  & 0     & 90  & r & -90 & 90 & 180 & 360 \\
			90  & 0     & 9.5   & 0   & r & -90 & 90 & 180 & 360 \\
			\bottomrule
		\end{tabular}
		\caption{Parámetros DH del robot \texttt{robot0}. Todos los ángulos están en grados y las longitudes en centimetros.}
	\end{table}
	
	\textbf{Descripción de cada parámetro:}
	\begin{itemize}
		\item $\theta$: Ángulo de rotación en torno al eje z del eslabón anterior.
		\item $d$: Desplazamiento a lo largo del eje z del eslabón anterior.
		\item $a$: Longitud del eslabón, medida en la dirección del eje x.
		\item $\alpha$: Ángulo entre los ejes z consecutivos, medido alrededor del eje x.
		\item \textbf{tipo}: Indica si la articulación es rotacional (\texttt{r}).
		\item \textbf{min} y \textbf{max}: Límites del ángulo de cada articulación en grados.
		\item $\dot{q}_{\text{max}}$: Velocidad angular máxima permitida por articulación (°/s).
		\item $\ddot{q}_{\text{max}}$: Aceleración angular máxima permitida por articulación (°/s²).
	\end{itemize}
	
	\textbf{Interpretación:}  
	Este conjunto de parámetros define completamente la estructura geométrica y las limitaciones cinemáticas del robot. La uniformidad de los valores de velocidad y aceleración máxima indica que el robot fue modelado con restricciones homogéneas en cada articulación, lo cual es útil para simulaciones o análisis de control.
	


