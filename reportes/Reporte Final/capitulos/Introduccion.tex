\chapter{Introducción} \label{chap:introduccion}

La robótica moderna se fundamenta en el estudio detallado del movimiento, la interacción con el entorno y el control preciso de sistemas mecánicos complejos. En este contexto, la cinemática y la dinámica constituyen pilares fundamentales para modelar, analizar y controlar el comportamiento de los robots manipuladores, mientras que herramientas como Robot Operating System (ROS) permiten simular, visualizar y comunicar estos modelos dentro de entornos virtuales y reales.

El estudio de la \textbf{cinemática} proporciona los medios para describir el movimiento del robot sin considerar las fuerzas que lo generan. A través del análisis directo, diferencial e inverso, es posible predecir y controlar la posición, orientación y velocidad del efector final a partir de las variables articulares. Esta capacidad es esencial para tareas de posicionamiento, seguimiento de trayectorias y planificación del movimiento.

Por otro lado, la \textbf{dinámica} permite comprender cómo las fuerzas y torques influyen en el movimiento del sistema, incorporando factores como la masa, la inercia, la fricción y la gravedad. El modelado dinámico, a través de formulaciones como la de Euler-Lagrange, resulta indispensable para el diseño de controladores robustos y eficientes que garanticen un comportamiento estable y preciso bajo diferentes condiciones de operación.

Complementando estos aspectos teóricos, el uso de \textbf{ROS} y sus herramientas asociadas, como RViz y Gazebo, proporciona un entorno flexible y modular para la simulación, visualización y comunicación entre los diferentes componentes del sistema robótico. Estas herramientas permiten verificar algoritmos de control y percepción en escenarios virtuales antes de implementarlos en hardware real, reduciendo riesgos y acelerando el proceso de desarrollo.

Este reporte presenta una síntesis de estos conceptos clave, abordando desde el modelado cinemático y dinámico hasta la implementación práctica mediante simuladores.