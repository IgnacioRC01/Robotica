\chapter{Resultados} \label{chap:resultados}
\textbf{Cinematica inversa}

\begin{figure} [h]
	\centering
	\includegraphics[width=0.7\linewidth]{"../cinematica inversa/img/graficas_taryectoria 2"}
	\caption{Cinematica inversa}
	\label{fig:graficastaryectoria-2}
\end{figure}

\begin{figure} [h]
	\centering
	\includegraphics[width=0.7\linewidth]{"../cinematica inversa/img/cinematica articular_trayectoria2"}
	\caption{Cinemática inversa}
	\label{fig:cinematica-articulartrayectoria2}
\end{figure}
\newpage

\subsection{Cinemática Articular}
La Figura muestra la evolución temporal de las variables articulares del robot, correspondientes a cuatro grados de libertad. En la primera fila se grafican la posición angular $\theta$, velocidad angular $\dot{\theta}$ y aceleración angular $\ddot{\theta}$ de tres articulaciones rotacionales. En la segunda fila, se presenta la posición lineal $d$, velocidad lineal $\dot{d}$ y aceleración lineal $\ddot{d}$ de una articulación prismática. 

Se puede observar que las trayectorias están definidas mediante perfiles de velocidad trapezoidales, que permiten movimientos suaves y controlados. Las aceleraciones muestran transiciones bruscas entre fases de movimiento constante y fases de aceleración/desaceleración, lo cual es típico en sistemas con interpolación por segmentos lineales en el espacio de las articulaciones.


\subsection{Cinemática del Efector Final}
La Figura muestra la evolución de la posición, velocidad y aceleración del efector final tanto en coordenadas cartesianas como angulares. En la primera fila se observa que la posición en los ejes $X$ y $Y$ permanece constante, mientras que el eje $Z$ varía, lo que indica un movimiento predominantemente vertical. La velocidad lineal $v_z$ y la aceleración $a_z$ también reflejan este comportamiento.

En la segunda fila se presenta la orientación del efector final usando ángulos de Euler ($\phi$, $\theta$, $\psi$). En este caso, solo el ángulo $\psi$ cambia con el tiempo, lo que implica una rotación sobre el eje $Z$. Las gráficas de velocidad angular y aceleración angular muestran un perfil trapezoidal en $\dot{\psi}$ y una aceleración escalonada en $\ddot{\psi}$, típicos de un control de trayectoria planificado con perfiles suaves pero definidos.


