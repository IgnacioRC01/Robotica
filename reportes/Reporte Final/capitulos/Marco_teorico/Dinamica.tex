\section{Dinámica} \label{sec:dinamica}

La dinámica de un robot describe cómo las fuerzas aplicadas en sus articulaciones generan movimientos. A diferencia de la cinemática, la dinámica incluye efectos de masa, inercia y fuerzas externas.

El modelo dinámico estándar de un robot manipulador se expresa mediante la ecuación de Euler-Lagrange:

\begin{equation}
	\mathbf{M}(q)\ddot{q} + \mathbf{C}(q,\dot{q})\dot{q} + \mathbf{g}(q) = \boldsymbol{\tau}
\end{equation}

Donde:
\begin{itemize}
	\item \( q \in \mathbb{R}^n \): vector de coordenadas articulares.
	\item \( \dot{q} \): vector de velocidades articulares.
	\item \( \ddot{q} \): vector de aceleraciones articulares.
	\item \( \boldsymbol{\tau} \): vector de torques o fuerzas generadas por los actuadores.
	\item \( \mathbf{M}(q) \): matriz de masa o inercia.
	\item \( \mathbf{C}(q, \dot{q}) \): matriz de Coriolis y fuerzas centrífugas.
	\item \( \mathbf{g}(q) \): vector de gravedad.
\end{itemize}
 

\subsection{Matriz de masa o inercia}


La matriz de masa o inercia \( \mathbf{M}(q) \) describe cómo la masa del robot está distribuida en función de su configuración \( q \). Es una matriz simétrica y definida positiva, que relaciona los torques aplicados con las aceleraciones articulares:

\[
\ddot{q} = \mathbf{M}^{-1}(q) \boldsymbol{\tau}
\]

Para un mismo torque \( \boldsymbol{\tau} \), una matriz de inercia mayor implica una aceleración menor. Por lo tanto, \( \mathbf{M}(q) \) define los límites físicos del movimiento del robot y debe ser considerada en el diseño de controladores y trayectorias.


\subsection{Matriz de coriolis}

\(\mathbf{C}(q,\dot{q}) \)
Representa las fuerzas internas debidas al movimiento del robot, como los efectos de Coriolis y las fuerzas centrífugas. Se calcula usando los símbolos de Christoffel:

\[
C_{ij} = \sum_{k=1}^{n} \frac{1}{2} \left( \frac{\partial M_{ij}}{\partial q_k} + \frac{\partial M_{ik}}{\partial q_j} - \frac{\partial M_{jk}}{\partial q_i} \right) \dot{q}_k
\]


\subsection{Vector de gravedad}


El vector de gravedad \( \mathbf{g}(q) \) representa los torques necesarios para contrarrestar el peso de los eslabones del robot. Este vector alcanza su \textbf{valor máximo cuando el robot está completamente extendido en posición horizontal}, debido a que:

\begin{itemize}
	\item El torque gravitacional \( \tau = r \cdot m \cdot g \) es máximo cuando el ángulo entre el enlace y la dirección de la gravedad es de 90°.
	\item En esta configuración, cada articulación debe soportar su propio peso y el de todos los eslabones posteriores, generando una carga acumulativa.
\end{itemize}

Para mantener el robot estable en esta posición, los motores deben aplicar un torque igual y opuesto al generado por la gravedad. De no hacerlo, el robot cedería, afectando la precisión y el control. Por ello, esta condición es clave en el diseño y selección de actuadores.



\subsection{Fricción}


\subsection*{Fricción}

La fricción es una fuerza que se opone al movimiento relativo entre dos superficies en contacto. Es causada por las irregularidades microscópicas entre dichas superficies y por interacciones a nivel molecular.

En general, se distinguen dos tipos principales de fricción:


En los sistemas mecánicos, especialmente en robótica, es común modelar dos tipos principales de fricción que afectan el movimiento de las articulaciones: la fricción estática o seca, y la fricción dinámica o viscosa.


\subsubsection{Fricción estática o seca}

también conocida como fricción de Coulomb, es la fuerza que se opone al inicio del movimiento entre dos superficies en contacto. Tiene un valor máximo y constante que debe ser superado para que comience el deslizamiento. Esta fricción no depende de la velocidad, y se modela comúnmente como:
\[
\tau_{\text{seca}} = \tau_c \cdot \text{sign}(\dot{q})
\]
donde \( \tau_c \) es el valor constante de fricción de Coulomb, y \( \text{sign}(\dot{q}) \) indica la dirección del movimiento.

\subsubsection{Fricción dinámica o viscosa}


aparece cuando ya existe movimiento entre las superficies. Esta fricción es proporcional a la velocidad relativa entre las partes, y se modela como:
\[
\tau_{\text{viscosa}} = b \cdot \dot{q}
\]
donde \( b \) es el coeficiente de fricción viscosa, y \( \dot{q} \) es la velocidad de la articulación. Esta fricción es importante en movimientos continuos o rápidos.


\subsection{Perturbaciones}

En el contexto de sistemas mecánicos y de control, una perturbación es cualquier influencia externa o interna no deseada que afecta el comportamiento esperado del sistema. Las perturbaciones pueden ser causadas por factores como fuerzas externas, variaciones en la carga, errores de modelado, ruido en sensores o fricción no modelada.

Aunque no siempre se pueden eliminar, las perturbaciones deben considerarse al diseñar sistemas de control, ya que pueden afectar la precisión, estabilidad y desempeño del robot.
